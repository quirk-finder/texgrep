\documentclass{article}
\usepackage{amsmath,amssymb,mathtools}

\newcommand{\vect}[1]{\mathbf{#1}}
\newcommand{\norm}[1]{\lVert #1 \rVert}
\DeclareMathOperator{\argmin}{argmin}
\DeclareMathOperator{\argmax}{argmax}

\begin{document}
% Math showcase
Inline: $f(x)=\frac{1}{\sqrt{2\pi}}\mathrm{e}^{-x^2/2}$, $\norm{\vect{v}}_2$, $\iiint_V g\,\mathrm{d}V$.

Display:
\[
\nabla \cdot \vect{F}=\lim_{\varepsilon\to 0}\frac{1}{\varepsilon^3}
\iiint_{[-\varepsilon/2,\varepsilon/2]^3} \operatorname{div}\vect{F}\,\mathrm{d}V.
\]

Aligned:
\begin{align}
S_n &= \sum_{i=1}^n i = \frac{n(n+1)}{2}, \\
P_n &= \prod_{k=1}^n k = n!.
\end{align}

Sets: $\mathbb{R}, \mathbb{N}, \mathcal{O}(n)$, intervals $[a,b)$.

Limits: $\limsup_{n\to\infty} a_n$, $\argmax_{x\in\mathbb{R}} f(x)$.

% theorem 環境はレンダしないが検索用に置く
\begin{theorem}
If $f$ is continuous on $[a,b]$, then $\int_a^b f'(x)\,\mathrm{d}x=f(b)-f(a)$.
\end{theorem}

Labels/refs: See \eqref{eq:binom} and Eq.~\ref{eq:binom}.
\begin{equation}\label{eq:binom}
\binom{n}{k}=\frac{n!}{k!(n-k)!}.
\end{equation}

% Various commands to hit
\newcommand{\R}{\mathbb{R}}
\renewcommand{\vec}[1]{\underline{#1}}
\end{document}
